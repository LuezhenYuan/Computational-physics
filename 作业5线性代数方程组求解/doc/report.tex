\documentclass[10pt,a4paper]{article}
\usepackage[UTF8]{ctex}
\usepackage{tikz}  
\usepackage{amsmath}
\usepackage{amsfonts}
\usepackage{amssymb}
\usepackage{float}
%\usepackage{clrscode}
\usepackage[]{algorithm2e}
\usepackage{graphicx}
% Set the margin of document.
\usepackage[top=10mm, bottom=12.5mm, left=12.5mm, right=12.5mm]{geometry}
\usepackage{longtable}
\usepackage{float}
\usepackage{listings}
\lstset{
  basicstyle=\small,
  numbers=left,
  keywordstyle=\color{blue},
  numberstyle={\tiny\color{lightgray}},
  stepnumber=1, %行号会逐行往上递增
  numbersep=5pt,
  commentstyle=\small\color{red},
  backgroundcolor=\color[rgb]{0.95,1.0,1.0},
  showspaces=false,
  showtabs=false,
  frame=shadowbox, framexleftmargin=5mm, rulesepcolor=\color{red!20!green!20!blue!20!},
% frame=single,
%  TABframe=single,
  tabsize=4,
  breaklines=tr,
  extendedchars=false %这一条命令可以解决代码跨页时,章节标题,页眉等汉字不显示的问题
}
\renewcommand\figurename{图}
\renewcommand\tablename{表}
\title{线性代数方程组的求解}
\author{袁略真\\3130103964\\生物信息学\\浙江大学}
\begin{document}
\maketitle

\section{线性代数方程组的解法}
\subsection{解析法}
线性代数方程组的精确求解可使用高斯消去法. 高斯消去法通过不断消元,得到$x_n$的解,然后回代得到其余变量的解. 每次消元有如下通式:
\begin{equation}
\left\lbrace
\begin{aligned}
B_k^k =& \frac{B_k^{k-1}}{A_{kk}^{k-1}}\ k=1,..,n\\
A_{kj}^k =& \frac{A_{kj}^{k-1}}{A_{kk}^{k-1}},\\
A_{ij}^k =& A_{ij}^{k-1} - A_{ik}^{k-1}*A_{kj}^k,\ i,j=k+1,..,n\\
B_j^k =& B_j^{k-1} - A_{jk}^{k-1}*B_k^k\\
\end{aligned}
\right.
\end{equation}
\subsubsection{改进的列主元素消去法}
原始的列主元素消去法需要将系数矩阵A的行互换(如第k和第j行).这通常需要申请一片暂时的空间,进行3*(n-k+1)次赋值操作.

改进的方法是将第j行加到第k行上去(A(k,k)同号时加,异号时减).方程组的解不变.
\subsection{数值法}
数值法求解线性方程组可用塞得尔迭代法.相比于高斯消元法,迭代法实现起来更简单,并且可以处理非线性方程组.

迭代法需要给定`解'的初值,然后根据如下形式解反复迭代,直到收敛.若能收敛,则收敛到精确解.
$$x_i = \frac{B_{i}}{A_{ii}} - \frac{1}{A_{ii}} \sum\limits_{j=1,j\neq i}^n A_{ij}x_j,\ i=1,2,..,n$$

\section{求解结果}
高斯消元法(使用列主元素消去法与否结果相同):
$$\textbf{x} = [1,2,3,1]^T$$

塞得尔迭代法(迭代28次收敛,$\varepsilon = 0.00001$):
$$\textbf{x} = [1,2,2.99999,0.999982]^T$$
\end{document}